\documentclass[11pt]{article}

\usepackage[hyphens]{url}
\usepackage[pdftex,bookmarksnumbered,hidelinks,breaklinks]{hyperref}
\usepackage{amssymb}
\usepackage{amsmath}
\usepackage{amsrefs}
\usepackage{graphicx}
\usepackage{bookman}
\usepackage{amsthm}
\usepackage{verbatim}
\usepackage{xcolor}
\usepackage{setspace}
\usepackage{float}
\usepackage{url}
\usepackage[margin=1in]{geometry}
\bibliographystyle{amsmath}
\newtheorem{definition}{Definition}
\newtheorem{theorem}{Theorem}
\newcommand{\?}{\stackrel{?}{=}}
\begin{document}

\title{CPSC 8100 Project Proposal}
\date{}
\author{Tyler Allen}
\maketitle

\doublespacing

\section{Project Background}
The chosen project is "Right Whale Recognition"\cite{kaggle_desc}. The objective of
this project is to develop software that will accept photographic input, and 
output the identification number associated with the whale as well as the probability
that this matching is correct. This will be accomplished using machine learning 
and facial recognition techniques. 

\section{Motivation}
This project provides a complex challenge for machine learning. The program
must be able to learn how to identify a whale in a photograph. The program must
be precise enough to differentiate between different whales. It must also be efficient, due to the
size of the photographic data set. The hypothesis space for photographs is also
much less obvious than a standard text data set. Therefore, the data set introduces real-world performance
constraints. In addition, the data set is provided by the National Oceanic and 
Atmospheric Administration (NOAA) and the New England Aquarium (NEA), so this is a real-world problem that will 
prove to be sufficiently challenging\cite{kaggle_desc}\cite{NOAA}\cite{NEA}.
 
 \newpage
\section{Development}

\subsection{Data Set}
The project host has provided a data set\cite{kaggle_data}. The data set includes 
$11469$ photos. There is also a set of training data, with $4544$ training examples.
This data set is sufficent for training and testing machine learning algorithms. 

\subsection{Facial Recognition}
For this competition, MathWorks is providing their MatLab software, including
their "Training Image Labeler app"\cite{kaggle_face}\cite{math_training}. This 
software can be trained and used with different machine learning techniques\cite{kaggle_face}\cite{math_cascade}. 
This will be the primary facial recognition tool used in this project. If time
permits, other methods will be investigated and implemented.

\subsection{Machine Learning Techniques}
A variety of machine learning algorithms covered in class and researched independently
will be implemented and the results compared. This is necessary because the most 
accurate algorithm may not be ideal for this problem. The data set is large and 
complex, so the complexity of more accurate algorithms may prove to be too 
computationally expensive. Some algorithms may later have to be ruled out 
if errors are discovered in the training data. 

\subsection{Implementation}
This project will be written in Java. This is to allow for an object oriented
and modular design. This will ease the ability to implement and use multiple
machine learning algoirthms and collect statistics. The MathWorks Training
Image Labeler App will be utilized through external calls\cite{kaggle_face}. This 
will also allow the facial recognition software to be modular to support 
other approaches to this problem. Multiple implementations using this modular 
design will provide a multitude of result sets for comparison and analysis.

\section{Timeline}
\begin{itemize}
\item[~]
\textbf{First Progress Report}\\
Program will be able to make use of the data set. External facial recognition 
interface will be functional. Basic machine learning algorithm will be completed.
\item[~]
\textbf{Second Progress Report}\\
Implementation of more advanced machine learning algorithms. Additional facial
recognition or other approaches will be implemented.
\item[~]
\textbf{Third Progress Report}\\
Any remaining algorithms or approaches will be implemented. Analysis for comparing
algorithms will be complete. Ideal algorithm will be chosen for competition submission.
\end{itemize}

\section{Conclusion}
Due to the data set being in the form of photographs, the solutions will pull 
heavily on machine learning theory and topics. This project will also require 
comparison and analysis work to contrast different solutions. In conclusion, 
this project is a challenging real-world problem for machine learning.

\newpage
\bibliography{proposal}
\nocite{*}
\end{document}
